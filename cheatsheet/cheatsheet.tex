% ----------------------------------------------------------------------------------------------------------------------
% RUST cheat sheet - Released under the MIT License
% ----------------------------------------------------------------------------------------------------------------------

% User guides
% Latex: https://ftp.agdsn.de/pub/mirrors/latex/dante/macros/latex/base/usrguide.pdf
% Tabularray: https://mirror.physik.tu-berlin.de/pub/CTAN/macros/latex/contrib/tabularray/tabularray.pdf
% TColorbox: https://mirror.clientvps.com/CTAN/macros/latex/contrib/tcolorbox/tcolorbox.pdf
% Posterbox Tutorial: https://mirror.clientvps.com/CTAN/macros/latex/contrib/tcolorbox/tcolorbox-tutorial-poster.pdf#[0,{%22name%22:%22Fit%22}]
% Xcolor: https://mirror.clientvps.com/CTAN/macros/latex/contrib/xcolor/xcolor.pdf
% TikZ: https://tikz.dev/

\documentclass[8pt]{extarticle}


% --- Hyphenation Rules --------------------------------------------------------
\usepackage[USenglish]{babel}

% --- Fonts --------------------------------------------------------------------

% Use a modern, Arial-like sans-serif as the document default (Helvetica
% substitute). This is PDFLaTeX-safe. If you prefer true Arial and are
% compiling with XeLaTeX/LuaLaTeX, switch to fontspec instead and load
% the system Arial.
\usepackage[scaled]{helvet}
\renewcommand{\familydefault}{\sfdefault}

% font for the word "RUST" in the header
\usepackage{alfaslabone}

\usepackage[T1]{fontenc}


% --- Page layout --------------------------------------------------------------
\usepackage[a4paper,landscape, right=2.5mm, left=2.5mm, top=2.5mm, bottom=2.5mm]{geometry}

% --- Listings -----------------------------------------------------------------

% Default minted options: break long lines, small font, trim leading space,
% and use a light gray background for code blocks. Per-instance options can
% still override these (for example: \begin{minted}[bgcolor=white]{rust}).
% \setminted{breaklines=true, fontsize=\scriptsize, autogobble=true, bgcolor=gray!8}

% --- TColorbox ---------------------------------------------------------------
\usepackage{tcolorbox}
\tcbuselibrary{skins,raster,listings,breakable,minted, poster}


% --- Tables -------------------------------------------------------------------
\usepackage{tabularray}

% --- Color Names -------------------------------------------------------------
\usepackage[dvipsnames*,svgnames]{xcolor}

% --- Graphics -----------------------------------------------------------------
\usepackage{tikz}
\usetikzlibrary {chains} 
\usepackage{graphicx}
\usepackage{svg}
\graphicspath{{./figures/}{./icons/}{./logos/}}

% --- URL and href -----------------------------------------------------------
\colorlet{citecolor}{black}
\colorlet{linkcolor}{black}
\colorlet{urlcolor}{black}
\usepackage[
  bookmarks=true,
  breaklinks=true,
  pdfborder={0 0 0},
  citecolor=citecolor,
  linkcolor=linkcolor,
  urlcolor=urlcolor,
  colorlinks=true,
  linktocpage=false,
  hyperindex=true,
  colorlinks=true,
  linktocpage=false,
  linkbordercolor=white]{hyperref}

% --- Helpers -----------------------------------------------------------------
\usepackage{lipsum}
\usepackage{enumitem}

% --- Default options ---------------------------------------------------------
\pagestyle{empty}
% \setlength\parindent{0pt}
% \setlength{\tabcolsep}{2pt}
% \baselineskip=0pt
% \setlength\columnsep{1.75mm}
% \setlength{\parskip}{0.1\baselineskip}



% --- Macros ------------------------------------------------------------------

% Custom command for adding a shorcut, like: "CTRL+C ........ Copy"
\newcommand{\command}[2]{#1~\dotfill{}~#2\\} 

% Colorbox environment for bash listings
\newtcblisting{bashlisting}
    {
        size=minimal,
        listing engine=minted,
        listing only,
        minted style=friendly,
        minted language=bash,
        minted options={
            fontsize=\footnotesize,
            breaklines,
            autogobble,
        },
        colback=gray!10!white,
        colframe=gray!10!white,
        listing only,
        left=0em,
        enhanced,
        % overlay={
        %     \begin{tcbclipinterior}
        %         \fill[red!20!blue!20!white] (frame.south west)
        %         rectangle ([xshift=5mm]frame.north west);
        %     \end{tcbclipinterior}
        % }
    }


% Skin for header box
\tcbsubskin{headerboxskin}{empty}{
        size=minimal,
        coltitle=black!10!black,
}


% Skin for orange-based cheat section boxes
\tcbsubskin{cheatboxorangeskin}{standard}{
    size=title, 
    left=0.2em,
    right=0.2em,
    arc=0.25mm,
    colback=orange!2!white,
    colframe=orange!75!black,
    coltitle=orange!10!black,
    colbacktitle=orange!75!white,
    fonttitle=\sffamily\bfseries,
    toptitle=0mm,
    bottomtitle=0mm,
    fontupper=\sffamily\small,
    fontlower=\sffamily\small,
    halign=left,
    subtitle style={top=0.4mm, bottom=0mm, boxrule=0.2pt, boxsep=0.1mm,
        colback=orange!50!orange!20!white,
        colupper=orange!50!gray,
        fontupper=\sffamily\bfseries\small,
        coltext=orange!10!black,
        height=1.1\baselineskip,
      }, 
}


% Skin for simple tables inside cheat boxes
\tcbsubskin{cheatboxtablesimple}{empty}{
    size=minimal,
    before skip=0.2\baselineskip,
    after skip=0.2\baselineskip,
    colback=white,
    colframe=gray,
    frame empty,
}



\NewDocumentCommand{\memorylayout}{O{red} m m !O{}}
{
    \begin{tcboxedraster}
        [
            raster columns=8,
            raster before skip=0pt,
            size=tight,
            raster column skip=1pt,
            raster row skip=1pt,
            raster left skip=1pt, 
            raster right skip = 1pt,
            top=0.4ex,
            bottom=0.4ex,
            halign=center,
            colframe=#1!40!gray,
            colback=#1!10,
        ]
        {
            skin=spartan,
            size=minimal,
            nobeforeafter,
            before skip balanced=-0.9\baselineskip,
            width=#3,
            colback=gray!20!white,
            colframe=gray
        }%
            \foreach \x in {1,...,#2}%
            {%
                \begin{tcolorbox}[coltext=gray]\tiny\x\end{tcolorbox}%
            }%
    \end{tcboxedraster}
}

% \NewDocumentCommand{\memorylayout}{O{red} m m !O{}}
% {
%     \begin{tcolorbox}
%     [
%         size=tight,
%         nobeforeafter, 
%         before skip balanced=-1\baselineskip,
%         % top=0.2ex,
%         % bottom=0.2ex,
%         % left=0.02em,
%         % right=0.02em,
%         width=#3,
%         arc=1pt,
%         outer arc=1pt,
%         coltext=black,
%         colback=#1!10,
%         colframe=#1!40!gray
%     ]
%     {
%         \begin{tikzpicture}[start chain]
%         % The chain is called just "chain"
%         \node [on chain] {A};
%         \node [on chain] {B};
%         \node [on chain] {C};
%         \end{tikzpicture}
%     }
%     \end{tcolorbox}
% }


% \tikzstyle{terminator} = [rectangle, draw, text centered, rounded corners, minimum height=2em]


% {skin=spartan,size=tight,before skip=0pt,width=#3,arc=1pt,outer arc=1pt,colback=#1!10,colframe=#1!20!gray,fonttitle=\bfseries}


% \NewDocumentCommand{\memorylayout}{O{red} m m !O{}}
% {
%     \begin{tcboxedraster}
%     [
%         size=minimal,
%         colback=orange!60!white,
%         colframe=orange!50!black,
%         raster before skip=0pt, 
%         raster after skip=0pt,
%         raster column skip=0.2em,
%         raster columns=#2, 
%         raster width=#3, 
%         raster height=6ex, 
%         raster equal height,
%         raster every box/.style=
%             {
%                 size=fbox,
%                 colframe=red!50!black,
%                 colback=red!10!white,
%                 valign=center,
%                 halign=center
%             }
%     ]
%         \foreach \x in {1,...,#2}
%         {
%             \begin{tcolorbox}
%                 \x
%             \end{tcolorbox}
%         }
%     \end{tcboxedraster}
% } 

% \NewTotalTCBox[]{\memorylayout}{ O{red} m m !O{} }
% {
%     size=fbox,
%     arc=0.2em,
%     colback=orange!60!white,
%     colframe=orange!50!black,
%     on line,
% }
% {
%      \begin{tcbraster}[size=minimal,raster before skip=0pt, raster after skip=0pt,raster column skip=0.2em,raster columns=#2, raster width=#3, raster height=6ex, raster equal height,
%         raster every box/.style={size=fbox,colframe=red!50!black,colback=red!10!white,
%             valign=center,halign=center}]
        
%         \foreach \x in {1,...,#2}{%
%             \begin{tcolorbox}
%                 \x
%             \end{tcolorbox}
%         }
    
%     \end{tcbraster}

% }


    
% ----------------------------------------------------------------------------------------------------------------------
% Main Document
% ----------------------------------------------------------------------------------------------------------------------

\begin{document}

        \begin{tcolorbox}
            [
                % skin=cheatboxtablesimple,
                tabularray={
                        hline{1-Z} = {1pt,solid},
                        % columns={halign=l, valign=m},
                    }
            ]
            \texttt{bool}   clkw ecwecw pec weco woec woec woec woe         & 		\begin{tcolorbox}[skin=spartan,size=minimal,nobeforeafter,width=15em,coltext=black,colback=lightgray]This is a \textbf{tcolorbox}.\end{tcolorbox}   & Boolean \texttt{true} or \texttt{false} \\
            \texttt{bool}   clkw ecwecw pec weco woec woec woec woe         & 		\begin{tcboxedraster}[raster columns=8,raster before skip=0pt,size=tight,raster column skip=1pt,raster left skip=1pt, raster right skip = 1pt,top=0.4ex,bottom=0.4ex,halign=center,colframe=gray!50!black,colback=gray!10!white,colbacktitle=gray!50!white]{skin=spartan,size=minimal,nobeforeafter,width=15em,coltext=black,colback=lightgray}\foreach \x in {1,...,8}{\begin{tcolorbox}[coltext=green!20!gray]\tiny\x\end{tcolorbox}}\end{tcboxedraster}   & Boolean \texttt{true} or \texttt{false} \\
            \texttt{bool}   clkw ecwecw pec weco woec woec woec woe         & 		\memorylayout[green!70!white]{8}{9em}[]    & Boolean \texttt{true} or \texttt{false} \\
        \end{tcolorbox}

\memorylayout[green!70!white]{8}{9em}[]  


\pagebreak

% === Page 1 =================================================================

\begin{tcbposter}[
        poster = {spacing=0.6em, columns=12},
        boxes = {},
    ]

    % === Header ================================================================
    \begin{posterboxenv}[skin=headerboxskin]
        {name=headertitle, span=3, below=top}
        \LARGE\raisebox{-0.1\height}{\includesvg[height=0.8\baselineskip]{logos/rust-logo-blk}}~{\fontfamily{AlphaSlabOne-TLF}\selectfont RUST} \textcolor{orange}{\mdseries for beginners}
    \end{posterboxenv}


    % === Overview ================================================================
    \begin{posterboxenv}[skin=cheatboxorangeskin, adjusted title=Overview]
        {name=overview,column=1,below=headertitle, span=3}
        Rust is a general-purpose programming language focused on performance, memory safety without garbage collector, and concurrency, using an ownership system preventing memory errors.

        \tcbsubtitle{\href{https://cheats.rs/\#hello-rust}{Strengths}}
        \begin{itemize}[nosep,leftmargin=1.5em]
            \item Compiled code about same performance as C / C++, and excellent memory and energy efficiency.
            \item Can avoid 70\% of all safety issues present in C / C++, and most memory issues.
            \item Strong type system prevents data races, brings 'fearless concurrency' (amongst others).
            \item Seamless C interop, dozens of supported platforms (LLVM).
        \end{itemize}

        \tcbsubtitle{\href{https://rustup.rs/}{Installation}}
        \begin{bashlisting}
            curl --proto '=https' --tlsv1.2 -sSf https://sh.rustup.rs | sh
        \end{bashlisting}
    \end{posterboxenv}


    % === Tools ================================================================
    \begin{posterboxenv}[skin=cheatboxorangeskin, adjusted title=Tools]
        {name=tools,column=1,below=overview, span=3}
        \tcbsubtitle{Rustup}
        {
            Install and manage Rust toolchains (compiler and tools).
            \begin{tcolorbox}
                [
                    skin=cheatboxtablesimple,
                    tabularray={
                            columns={halign=l, valign=t},
                            column{1}={colsep=0pt},
                            column{2}={15em, leftsep=1em, rightsep=0pt},
                            rows={rowsep=0pt},
                        }
                ]
                \texttt{rustup show}               & Show installed toolchains                             \\
                \texttt{rustup update}             & Update all toolchains                                 \\
                \texttt{rustup default TC}         & Set the default toolchain to \texttt{TC}              \\
                \texttt{rustup component list}     & List available components                             \\
                \texttt{rustup component add C}    & Add component \texttt{C}                           \\
                \texttt{rustup target list}        & List available compilation targets                    \\
                \texttt{rustup target add T}       & Add a compilation target \texttt{T}                \\
                \texttt{rustup docs}               & Open local Rust docs in browser \\
                \texttt{rustup uninstall TC}       & Uninstall toolchain \texttt{TC}                       \\
            \end{tcolorbox}
        }

        \tcbsubtitle{\href{https://rust-lang.github.io/rustup/concepts/components.html}{Components}}
        {
            Parts of toolchains, partly \textit{man}datory partly \textit{opt}ional.
            \begin{tcolorbox}
                [
                    skin=cheatboxtablesimple,
                    tabularray={
                            % hline{2} = {0.5pt,solid},
                            columns={halign=l, valign=m},
                            column{1}={colsep=0pt},
                            column{2}={leftsep=0.5em, rightsep=0pt},
                            column{3}={17.75em, leftsep=0.5em, rightsep=0pt},
                            rows={rowsep=0pt},
                        }
                ]
                \texttt{rustc}         & \textit{man} & The Rust compiler and \texttt{Rustdoc}                                                     \\
                \texttt{cargo}         & \textit{man} & Package manager and build tool                                                             \\
                \texttt{rustfmt}       & \textit{opt} & Code formatter                                                                             \\
                \texttt{rust-std}      & \textit{man} & Rust standard library for each target                                                \\
                \texttt{rust-docs}     & \textit{opt} & Local copy of the Rust documentation                                                       \\
                \texttt{rust-analyzer} & \textit{opt} & Language server for editors and IDEs                                                       \\
                \texttt{clippy}        & \textit{opt} & Lint tool providing extra checks for common mistakes and stylistic choices                 \\
                \texttt{miri}          & \textit{opt} & Experimental Rust interpreter                                                              \\
                \texttt{rust-src}      & \textit{opt} & Local copy of standard library source                                                      \\
                \texttt{rust-mingw}    & \textit{opt} & Linker and platform libraries for building on \textit{x86\_64-pc-windows-gnu}  \\
                \texttt{llvm-tools}    & \textit{opt} & Collection of LLVM tools. Unstable                                                         \\
                \texttt{rustc-dev}     & \textit{opt} & Compiler as a library, only needed for development of tools linking to compiler      \\
            \end{tcolorbox}
        }

        \tcbsubtitle{Cargo}
        {
            Used to build and run Rust projects..
            \begin{tcolorbox}
                [
                    skin=cheatboxtablesimple,
                    tabularray={
                            % hline{2} = {0.5pt,solid},
                            columns={halign=l, valign=t},
                            column{1}={colsep=0pt},
                            column{2}={15em, leftsep=0.5em, rightsep=0pt},
                            rows={rowsep=0pt},
                        }
                ]
                \texttt{cargo init}                & Create a new binary project                                   \\
                \texttt{cargo init --lib}          & Create a new library project                                  \\
                \texttt{cargo check}               & Check code for errors                                         \\
                \texttt{cargo clippy}              & Lint code                                                     \\
                \texttt{cargo doc}                 & Generate project documentation                                        \\
                \texttt{cargo run}                 & Run the project                                               \\
                \texttt{cargo run --bin NAME}      & Run a specific project binary                                 \\
                \texttt{cargo build}               & Build everything in debug mode                                \\
                \texttt{cargo build --bin NAME}    & Build binary \texttt{NAME} in debug mode                               \\
                \texttt{cargo build --release}     & Build everything in release mode                              \\
                \texttt{cargo build --target NAME} & Build for a specific target \texttt{NAME}                                  \\
                \texttt{cargo --explain CODE}      & Details on compiler error \texttt{CODE}                                \\
                \texttt{cargo test}                & Run all tests                                                 \\
                \texttt{cargo test TEST\_NAME}     & Run a specific test \texttt{TEST\_NAME}                                          \\
                \texttt{cargo test --doc}          & Run doctests only                                             \\
                \texttt{cargo bench}               & Run benchmarks
            \end{tcolorbox}
        }

    \end{posterboxenv}


    % === Primitive Data Types ================================================================
    \begin{posterboxenv}[skin=cheatboxorangeskin, adjusted title=Primitive Data Types]
        {name=primitives,column=4,below=headertitle, span=4}

        \tcbsubtitle{\href{https://cheats.rs/\#memory-layout}{Boolean and Fixed Point Numeric Types}}
        \begin{tcolorbox}
            [
                skin=cheatboxtablesimple,
                tabularray={
                        hline{1-Z} = {1pt,solid},
                        columns={halign=l, valign=m},
                        column{1}={colsep=0pt},
                        column{2}={leftsep=0.1em, rightsep=0pt},
                        column{3}={15em, leftsep=0.3em, rightsep=0pt},
                        rows={rowsep=0pt},
                    }
            ]
            \texttt{bool}           & 		\memorylayout[gray!70!white]{1}{9em}[]   & Boolean \texttt{true} or \texttt{false} \\
            \texttt{u8,i8}          & 		\memorylayout[green!70!white]{1}{9em}[]   & \texttt{0..255}, \texttt{-128..127} \\
            \texttt{u16,i16}        & 		\memorylayout[green!70!white]{2}{9em}[]   & \texttt{0..65535}, \texttt{-32768..32767} \\
            \texttt{u32,i32}        & 		\memorylayout[green!70!white]{4}{9em}[]   & \texttt{0..4294967295}, \texttt{-2147483648..2147483647} \\
            \texttt{u64,i64}        & 		\memorylayout[green!70!white]{8}{9em}[]   & \texttt{0..18446744073709551615}, \tiny{\texttt{-9223372036854775808.. 9223372036854775807}} \\
            \texttt{u128,i128}      & 		\memorylayout[green!70!white]{16}{9em}[]   & \tiny{\texttt{0..340282366920938463463374607431768211455} \texttt{-170141183460469231731687303715884105728.. 170141183460469231731687303715884105727}} \\
            \texttt{usize,isize}    & 		X   & Same as \texttt{ptr} on platform.  \\
        \end{tcolorbox}

        \tcbsubtitle{\href{https://cheats.rs/\#memory-layout}{Floating Point Numeric Types}}
        \begin{tcolorbox}
            [
                skin=cheatboxtablesimple,
                tabularray={
                        % hline{1-Z} = {1pt,solid},
                        columns={halign=l, valign=m},
                        column{1}={colsep=0pt},
                        column{2}={leftsep=0.5em, rightsep=0pt},
                        column{3}={15em, leftsep=1em, rightsep=0pt},
                        rows={rowsep=0pt},
                    }
            ]
            \texttt{f16}            & 		\memorylayout[magenta!70!white]{2}{8em}[]  & 16-bit floating point \\
        \end{tcolorbox}

    \end{posterboxenv}



\end{tcbposter}

\pagebreak




\end{document}
